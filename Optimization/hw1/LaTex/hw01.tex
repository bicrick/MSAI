\documentclass{article}
\usepackage{fancyhdr}
\usepackage{geometry}
\usepackage{amsfonts}
\usepackage{amsthm}

% Set page margins
\geometry{margin=1in}

% Define fancy header
\pagestyle{fancy}
\fancyhf{}
\renewcommand{\headrulewidth}{0.4pt}
\fancyhead[L]{Patrick Brown}
\fancyhead[C]{Optimization - Homework 1}
\fancyhead[R]{\thepage}

% Define a new environment for answers
\newenvironment{answer}
    {\par\noindent\textbf{Answer:}\par}
    {\par}

\begin{document}

\begin{center}
\Large\textbf{The University of Texas at Austin}\\[0.5em]
\large\textbf{Optimization}\\[0.5em]
\large\textbf{Homework 1}
\end{center}

\vspace{1em}

\textbf{Instructors:} Constantine Caramanis, Sujay Sanghavi

\vspace{1em}

\textbf{Submitting solutions:} Please submit your solutions as a single pdf file. If you have code or figures, please include these in the pdf.

\section{Convex Sets, Convex Functions, Preservation of Convexity}

\begin{enumerate}
    \item 
    \begin{enumerate}
        \item 
        \begin{answer}
        To show that the intersection of convex sets is convex, let $C = C_1 \cap C_2$ where $C_1$ and $C_2$ are convex sets. Let $\mathbf{x}, \mathbf{y} \in C$. This means $\mathbf{x}, \mathbf{y} \in C_1$ and $\mathbf{x}, \mathbf{y} \in C_2$.

        For any $\lambda \in [0,1]$, consider the point $\lambda\mathbf{x} + (1-\lambda)\mathbf{y}$.
        
        Since $C_1$ is convex, $\lambda\mathbf{x} + (1-\lambda)\mathbf{y} \in C_1$.
        
        Since $C_2$ is convex, $\lambda\mathbf{x} + (1-\lambda)\mathbf{y} \in C_2$.
        
        Therefore, $\lambda\mathbf{x} + (1-\lambda)\mathbf{y} \in C_1 \cap C_2 = C$.
        
        This proves that $C$ is convex, and thus the intersection of convex sets is convex.
        \end{answer}

        \item 
        \begin{answer}
        An example where the union of two convex sets is not convex:
        
        Consider two disjoint circles in $\mathbb{R}^2$. Each circle is a convex set, but their union is not convex because a line segment connecting a point in one circle to a point in the other circle would not be entirely contained within the union.
        \end{answer}

        \item 
        \begin{answer}
        To show that the maximum of convex functions is convex, let $f_1$ and $f_2$ be convex functions, and $f_{\max}(x) = \max\{f_1(x), f_2(x)\}$.
        
        For any $\mathbf{x}, \mathbf{y}$ and $\lambda \in [0,1]$:
        
        $f_{\max}(\lambda\mathbf{x} + (1-\lambda)\mathbf{y})$
        $= \max\{f_1(\lambda\mathbf{x} + (1-\lambda)\mathbf{y}), f_2(\lambda\mathbf{x} + (1-\lambda)\mathbf{y})\}$
        
        $\leq \max\{\lambda f_1(\mathbf{x}) + (1-\lambda)f_1(\mathbf{y}), \lambda f_2(\mathbf{x}) + (1-\lambda)f_2(\mathbf{y})\}$
        
        $\leq \lambda \max\{f_1(\mathbf{x}), f_2(\mathbf{x})\} + (1-\lambda) \max\{f_1(\mathbf{y}), f_2(\mathbf{y})\}$
        
        $= \lambda f_{\max}(\mathbf{x}) + (1-\lambda) f_{\max}(\mathbf{y})$
        
        This proves that $f_{\max}$ satisfies the definition of convexity, and thus the maximum of convex functions is convex.
        \end{answer}
    \end{enumerate}

    \section{More Convex Sets, Convex Functions, Preservation of Convexity}

\begin{enumerate}
    \item 
    \begin{enumerate}
        \item 
        \begin{answer}
        An example where the minimum of two convex functions is not convex:
        
        Consider $f_1(x) = x^2$ and $f_2(x) = (x-2)^2$. Both are convex functions.
        Let $f_{\min}(x) = \min\{f_1(x), f_2(x)\}$.
        
        $f_{\min}(x)$ is not convex because it forms a "V" shape with a non-convex kink at $x=1$, where the two parabolas intersect.
        \end{answer}

        \item 
        \begin{answer}
        An example of two closed convex sets that are disjoint but cannot be strictly separated:
        
        Consider in $\mathbb{R}^2$:
        $C_1 = \{(x,y) : y \geq e^x\}$
        $C_2 = \{(x,y) : y \leq 0\}$
        
        These sets are closed, convex, and disjoint. However, they cannot be strictly separated because for any separating hyperplane (in this case, a line), there will always be points from both sets arbitrarily close to the hyperplane as $x \to -\infty$.
        \end{answer}

        \item 
        \begin{answer}
        To show that any sub-level set of a convex function is convex:
        
        Let $f$ be a convex function and $L_c = \{x : f(x) \leq c\}$ be a sub-level set.
        Take any $x_1, x_2 \in L_c$ and $\lambda \in [0,1]$.
        
        $f(\lambda x_1 + (1-\lambda)x_2) \leq \lambda f(x_1) + (1-\lambda)f(x_2)$ (by convexity of $f$)
        $\leq \lambda c + (1-\lambda)c = c$
        
        Therefore, $\lambda x_1 + (1-\lambda)x_2 \in L_c$, proving $L_c$ is convex.
        
        Example of a non-convex function with convex sub-level sets:
        
        Consider $f(x) = -x^2$. This function is concave (thus not convex), but all its sub-level sets are convex intervals. For any $c$, $L_c = \{x : -x^2 \leq c\} = [-\sqrt{-c}, \sqrt{-c}]$, which is a convex set.
        \end{answer}
    \end{enumerate}
\end{enumerate}


\end{enumerate}

\section{Half-Space Representation of Points Closer to v1 than v2}

\begin{enumerate}
    \item Two dimensions:
    \begin{answer}
    For $v_1 = (-1, 0)^T$ and $v_2 = (1, 0)^T$, the set of points closer to $v_1$ than $v_2$ forms a half-space. This half-space is the left half of the plane, separated by the vertical line x = 0. The shaded region would be all points with x < 0.
    \end{answer}

    \item Finding c and d for the two-dimensional case:
    \begin{answer}
    For the given example, we need to find $c = (c_1, c_2)^T$ and $d$ such that:
    
    $\{(x_1, x_2)^T : c_1x_1 + c_2x_2 \leq d\}$
    
    represents the shaded region (x < 0).
    
    We can choose:
    $c = (1, 0)^T$ and $d = 0$
    
    This gives us the inequality: $x_1 \leq 0$, which correctly describes the left half-plane.
    \end{answer}

    \item Generalization to n dimensions:
    \begin{answer}
    For general points $v_1, v_2 \in \mathbb{R}^n$, we can find $c \in \mathbb{R}^n$ and $d \in \mathbb{R}$ as follows:
    
    $c = v_2 - v_1$
    $d = \frac{1}{2}(||v_2||^2 - ||v_1||^2)$
    
    To prove this, we start with the condition $||x - v_1||_2 \leq ||x - v_2||_2$:
    
    $(x - v_1)^T(x - v_1) \leq (x - v_2)^T(x - v_2)$
    
    Expanding and simplifying:
    
    $x^Tx - 2v_1^Tx + v_1^Tv_1 \leq x^Tx - 2v_2^Tx + v_2^Tv_2$
    
    $-2v_1^Tx + v_1^Tv_1 \leq -2v_2^Tx + v_2^Tv_2$
    
    $2(v_2 - v_1)^Tx \leq v_2^Tv_2 - v_1^Tv_1$
    
    $(v_2 - v_1)^Tx \leq \frac{1}{2}(v_2^Tv_2 - v_1^Tv_1)$
    
    This is equivalent to $c^Tx \leq d$ with $c$ and $d$ as defined above.
    
    Thus, we have shown that the set of points in $\mathbb{R}^n$ that are closer to point $v_1$ than to point $v_2$ indeed form a half-space, represented by $\{x : c^Tx \leq d\}$.
    \end{answer}
\end{enumerate}

\end{document}